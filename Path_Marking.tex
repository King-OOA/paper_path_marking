\documentclass{article}


\usepackage{enumerate}

\usepackage{graphicx}
\usepackage{algorithm}
\usepackage{algorithmic}
\usepackage{multirow}
\usepackage{amsmath}
\usepackage{color}
\newcommand{\SWITCH}[1]{\STATE \textbf{switch} (#1)}
\newcommand{\ENDPWITCH}{\STATE \textbf{end switch}}
\newcommand{\CASE}[1]{\STATE \textbf{case} #1\textbf{:} \begin{ALC@g}}
\newcommand{\ENDCASE}{\end{ALC@g}}
\newcommand{\CASELINE}[1]{\STATE \textbf{case} #1\textbf{:} }
\newcommand{\DEFAULT}{\STATE \textbf{default:} \begin{ALC@g}}
\newcommand{\ENDDEFAULT}{\end{ALC@g}}
\newcommand{\DEFAULTLINE}[1]{\STATE \textbf{default:} }
\newtheorem{mydef}{Definition}
\newtheorem{mylm}{Proposition}

\renewcommand{\algorithmicrequire}{\textbf{Input:}}
\renewcommand{\algorithmicensure}{\textbf{Output:}}


\begin{document}


\title{A Efficient Exact Algorithm for Multiple Longest Common
  Subsequences (MLCS) Problem Based on Path Marking}

\author{Zhan Peng, Yuping Wang\footnote{Corresponding author}~,}


\maketitle

%begin{history}
%received{(Day Month Year)}
%revised{(Day Month Year)}
%accepted{(Day Month Year)}
%comby{(xxxxxxxxxx)}
%end{history}

\begin{abstract}

  Searching for the Multiple Longest Common Subsequences (MLCS) of
  multiple sequences is a classical NP-hard problem that is widely
  used in many areas such as bioinformatics and biomedicine,
  etc. Although significant efforts have been made to address this
  problem, the increasing amount of biological data require more
  efficient methods. In this paper, we present a novel fast algorithm
  for the MLCS problem, which is based on the dominant point approach
  and employs a new technique called path marking. During the
  construction of the dominant point graph, the path marking technique
  can real-timely mark the longest path from the source node to the
  current node, and once the graph has been constructed the longest
  paths corresponding to every MLCS can be found immediately. The
  experimental results demonstrate that, comparing with the ``Minima''
  operation that is widely used in the existing algorithms, the path
  marking is much more efficient, and therefore our algorithm is
  faster than those algorithms.

\end{abstract}

\textbf{multiple longest common subsequences (MLCS); Longest common
  subsequence (LCS); dominant point method}

% due to the effectiveness of the filter and the compactness of the AMT
% whether there are patterns starting at that position

\section{Introduction}
\label{sec:introduction}


Many kinds of biological data can be represented as a sequence of
symbols. For instance, proteins [2] are sequences over 20 different
symbols (amino acids), and DNA (genes) can be represented as sequences
over four symbols A, C, G and T corresponding to the four bases.
Measuring the similarity of biological sequences is a fundamental
problem in bioinformatics, which can be used in many applications such
as cancer diagnoses, detection of the species’ common origin [2],
etc. One of the most important ways to measure the similarity of
biological sequences is to find their Multiple Longest Common
Subsequences (MLCS), which has been proved to be an NP-hard problem []
. Traditionally, the works are focusing on either the simplest case of
two sequences, also known as the Longest Common Subsequences (LCS)
problem [24], [36], [39], [45], or the problem’s special case of
three sequences [21], [22]. Although many algorithms have been
proposed for the general case of the problem, they are not very
efficient in practice for large-size problems with many sequences due
to high time and space complexity. Particularly, with the successful
implementation of the human genome project and the development of the
next-generation sequencing techniques, the amount of biological
sequences (as well as other types of sequences from various fields)
are growing explosively [3]. Therefore, it is very important and
valuable to design more efficient MLCS algorithm to process large
amount of biological (and other types) sequences.

In this paper, we proposed a fast algorithm for finding MLCS of many
sequences. Our algorithm is based on the dominant point model, in
which the dominant points produced during processing are organized to
form a Directed Acyclic Graph (DAG), and each of the longest paths
from the source node to the end node corresponds to a MLCS. Different
from the other dominant point based algorithms that using a
time-consuming \emph{Minima} operation to delete the redundant nodes,
and then search for the longest paths from the source node to the end
node to find the MLCS, our method employs a technical called
\emph{path marking}, whenever a node is generated, the longest path
from the source node to the new node is updated immediately, and once
the DAG is constructed, all the MLCS can be found directly from the
end node back to the source node by the marked path. The experiments
shows that the \emph{path marking} technique is much efficient than
the \emph{Minima} operation, so that our algorithm is much faster than
the compared algorithms.

The this paper is organized as follows: Section \ref{sec:related
  works} introduces some preliminaries and reviews the related works
in the MLCS problem. Section \ref{sec:notations} lists the notations
and terminology used in this paper. In section \ref{sec:algorithm},
the our new algorithm is presented in details. In section
\ref{sec:experiments}, experimental results comparing the performances
of the new algorithms with existing ones are presented. In Section
\ref{sec:conclusion} we summarize the paper.

\section{Preliminaries and Related Works}
\label{sec:related works}

In this section, we define some notations and terminology used in this
paper and reviews some related works in the MLCS problem.

\subsection{Definitions}
\label{sec:definitions}

Let $\Sigma$ represent a finite alphabet for the sequences.  For
example, for the DNA sequences, $\Sigma=\{A, C, T, G\}$.

\textbf{Definition 1.} Let $s=c_1c_2...c_n$ be a sequence with length
$n$ over $\Sigma$. A \emph{subsequence} of $s$ is the one that can be
obtained by deleting zero or more (not necessarily consecutive)
characters from $s$. Formally, $s'=c_{i_1}c_{i_2}...c_{i_k}$ is called
a subsequence of $s$, if for $1 \leq j \leq k$, there is
$1 \leq i_j \leq n$, and for all $r$ and $t$, $1 \leq r < t \leq k$,
there is $i_r < i_t$.\\

\textbf{Definition 2.} Let $S=\{s_1, s_2, ..., s_d\}$ be a set of $d$
sequences over $\Sigma$. A Longest Common Subsequence (\emph{LCS}) of
$S$ is a sequence $lcs$ such that: (1) $lcs$ is a subsequence of all
$s_i$ ($1 \leq i \leq d$); (2) there is no sequence satisfying (1) and
longer than $lcs$.\\

In general, given $d$ sequences, there usually exists more than one
LCS of the $d$ sequences. For example, given three sequences $s_1 =
ACTAGTGC$, $s_2 = TGCTAGCA$ and $s_3 = CATGCGAT$, there are two LCSs
with length 4, which are $lcs_1 = CAGC$, $lcs_2 = CTGC$,
respectively. The MLCS problem is to find all LCSs of three or more
sequences. For two sequences, the problem is commonly called the LCS
problem. Obviously, the LCS is a special case of the MLCS. Specific
algorithms developed for LCS are generally inefficient for MLCS.\\

\textbf{Definition 3.} Let $S = \{s_1,\, s_2,\, ...,\, s_d\}$ be a set
of $d$ sequences over $\Sigma$, and $s_i[j]$ be the $j$-th character
of $s_i$. A point $p = (p_1,\, p_2,\, ...,\, p_m)$ is called a
\emph{match point} of $S$, if $s_1[p_1] = s_2[p_2] = ... = s_d[p_d] =
\delta$, where $\delta$ is the symbol corresponding to the matched
point $p$.\\

\textbf{Definition 4.} Given two match points $p = (p_1,\, p_2,\,
...,\, p_d)$ and $q = (q_1,\, q_2,\, ...,\, q_d)$ of $S$, we say that
$p = q$, if and only if $p_i = q_i$ ($1 \leq i \leq d$). If $p_i \leq
q_i$ ($1 \leq i \leq d$), we say $p$ \emph{dominates} $q$, denoted as
$p \preceq q$. Further, if $p_i < q_i$ ($1 \leq i \leq d$), we say $p$
\emph{strongly dominates} $q$, denoted as $p \prec q$.\\

Because of the importance of PM, several efficient algorithms have
been proposed in the past decades.

\subsection{Dynamic Programming Based Algorithms}
\label{sec:DP}

The classical exact algorithms for MLCS problem are based on dynamic
programming \cite{Smith1981}, \cite{Sankoff1972}. Given $d$ sequences:
$s_1,\, s_2,\,...,\, s_d$ with length $n_1,\, n_2,\, ...,\, n_d$
respectively, these algorithms recursively construct a
$n_1 \times n_2 \times ... \times n_d$ score table $T$, in which
$T[i_1,\, i_2,\, ...,\, i_d]$ contains the length of the LCSs of the
prefixes $s_1[1...i_1]$, $s_2[1...i_2]$, ..., $s_d[1...i_d]$.
$T[i_1,\, i_2,\, ...,\, i_d]$ can be computed respectively by the
following formula:

\begin{equation*}
  T[i_1,\, i_2,\, ...,\, i_d] = 
  \begin{cases}
    0 & \text{if $\exists j, 1 \leq j \leq d, i_j = 0$}\\
    T[i_1-1,\, ...,\, i_d-1] + 1  & \text{if $s_1[i_1] = s_2[i_2] =
      ... = s_d[i_d]$}\\
    max(\bar{T}) & \text{otherwise}
  \end{cases}
\end{equation*}

where
$\bar{T} = \{T[i_1-1,\, i_2,\, ...,\, i_d],\, T[i_1,\, i_2-1,\, ...,\,
i_d],\, ...,\, T[i_1,\, i_2,\, ...,\, i_d-1]\}$. Once the score table
$T$ is constructed, all the MLCS can be collected by tracing back from
the last item $T[n_1,\, n_2,\, ...,\, n_d]$ to the first item
$T[0,\, 0,\, ...,\, 0]$. Fig. \ref{fig:DM} shows an example of score
table $T$ for two sequences $s_1 = GATTACA$ and $s_2 = GTAATCTAAC$.
Obviously, the time and space complexity of dynamic programming based
algorithms for a MLCS problem with $d$ sequences of length $n$ are
both $O(n^d)$ \cite{Hsu1984}, therefore, they can not be applied to
MLCS problems with more than two sequences in practice. To reduce the
complexity of dynamic programming based algorithms, various approaches
have been proposed \cite{Hirschberg1977}, \cite{Apostolico1992},
\cite{Masek1980}, \cite{Rick1994}. Unfortunately, these approaches
primarily address the LCS problem of two sequences .

\section{The Filter Module}
\label{sec:filter}

As mentioned before, our engine consists of a filter module and a
verification module. In this section, we introduce the former, while
in the next section the latter will be illustrated.

The filter we adopted in our engine is proposed by \cite{Lee2013},
which is an improvement of the filter in the WM algorithm. Given the
pattern set, as in the WM algorithm, only the length-$lsp$ prefix of
each pattern are considered in constructing the filter, where $lsp$ is
the \emph{length of the shortest pattern} in the pattern set. Similar
to WM, the filtering is based on character blocks (a character block
is merely a short string consisting of only a few characters) rather
than single character. Assuming that each block consists of $k$
characters, the filter will include $lsp-k+1$ membership query bitmaps
denoted as $B_1$, $B_2$, \dots, and $B_{lsp-k+1}$, where each bitmap
is a vector of $N$ bits. Let $B_j[i]$, $0 \leq i \leq N - 1$,
$1 \leq j \leq lsp-k+1$, be the $i$-th bit of the $j$-th query bitmap,
and $B_j[i]$ is set to 1 iff there exists a pattern $p_m$ such that
$hash(p_m[j,k]) = i$ ($p_m[j,k]$ is a $k$-symbol block starting at
position $j$ of $p_m$), where $hash$ is a hash function that maps a
$k$-symbol block to an integer in the range of $[0, N-1]$.

% \begin{figure}[htbp]
%   \centering
% ``  \includegraphics[width=0.6\textwidth]{./eps/filter}
%   \caption{The filter constructed from three digital patterns.}
%   \label{fig:filter}
% \end{figure}

Fig \ref{fig:filter} shows an example of the filter, assume that
alphabet $\Sigma = \{0, 1, 2, \dots, 9\}$ and pattern set
$P = \{p_1,\; p_2,\; p_3\}$, where $p_1 = 04648753$, $p_2 = 30692$,
and $p_3 = 614621$.  Obviously, $lsp = |p_2|= 5$. Assume that $k = 2$,
$N = 100$, and the hash function is the identity mapping. For such a
setting, there are $lsp - k + 1 = 4$ query bitmaps, i.e. $B_1$, $B_2$,
$B_3$, $B_4$, and in particular, we have $B_1[i] = 1$ iff $i$ = 4, 30,
or 61; $B_2[i] = 1$ iff $i$ = 6, 14, or 46; $B_3[i] = 1$ iff $i$ = 46,
64, or 69; and $B_4[i] = 1$ iff $i$ = 48, 62, or 92.

Once the query bitmaps have been constructed, they can be used to
filter the positions of the input text string. During filtering, a
sliding window $W$ of length $lsp$ is adopted to select a segment of
the text string $T$. Initially, $W$ is aligned with the beginning of
$T$, so that the substring of $T$ contained in $W$ is $T[1,lsp]$.
Generally, assume that $W$ is moved to the $i$-th position of $T$, and
the substring of $T$ contained in $W$ is $T[i,lsp]$. Then the last
$k$-symbol block of $T[i,lsp]$, which is $T[i+lsp-k, k]$, is used to
query all the bitmaps. Let single bit $qb_j$ be the report of $B_j$
($1 \leq j \leq lsp - k + 1$), and $qb_j=1$ iff
$B_j[hash(T[i+lsp-k,k])] = 1$. Let bitmap
$QB = qb_1qb_2 \dots qb_{lsp-k+1}$ to denote the current query result
and another \emph{master bitmap} $MB = mb_1mb_2 \dots mb_{lsp-k+1}$ to
accumulate the previous query results and act as the current state of
the filter. Initially, all the bits of $MB$ are set to 1. After
fetching a query result $QB$, the master bitmap is updated by
$MB = MB \; \& \; QB$, where \& is the bit-wised ``and''
operation. For the updated $MB$, if $mb_{lsp-k+1} = 1$, position $i$
is considered as a potential matched position and the verification
module is invoked. Otherwise, the sliding window $W$ is advanced by
$lsp-k+1$ positions if all the bits of $MB$ are 0's, or $lsp-k+1-r$
positions if $mb_r=1$ and $mb_i=0$, for $r < i \leq lsp-k$. The
positions skipped by $W$ are filtered out. If $W$ is determined to be
advanced by $h$ positions, $MB$ is right-shifted by $h$ bits and
filled with 1's for the holes left by the shift. Particularly, it is
proved in \cite{Lee2013} that the filter using master bitmap and
simple bit-wised ``and'' and ``shift'' operations is optimal in the
sense that it is able to utilize all previous query results, which
means the filter can filter out as many positions as possible in the
text.

% \begin{figure}[htbp]
%   \centering
%   \includegraphics[width=0.6\textwidth]{./eps/filter_match}
%   \caption{Filter the positions in the text using the filter.}
%   \label{fig:f_match}
% \end{figure}

As an example (see Fig \ref{fig:f_match}), consider the query bitmaps
constructed above and assume that $T=23764614621$.  Initially, the
matching window $W$ contains 23764 and $MB = 1111$.  The last 2-symbol
block of $W$, i.e. 64, is used for the first query, and the query
result is $QB=0010$. Since $MB\; \&\; QB = 0010$, $W$ is advanced by 1
position (which means the first position of $T$ is filtered out), and
then $MB$ is updated as $1001$ by right shift. For the second
position, the substring contained in $W$ is 37646. This time, block 46
is used to query the bitmaps and the query result is $QB=0110$. Since
$MB\; \& \; QB=0000$, $W$ moves forward by 4 positions to position 6
(which means positions $2 \sim 5$ are filled out), and $MB$ is updated
as 1111. For position 6, $W$ contains 61462 and block 62 is used for
querying and fetches $QB = 0001$. Since $MB\; \& \; QB = 0001$,
position 6 is a potential matched position, and thus, the verification
module is invoked. After checking, the pattern $p_3=614621$ is
detected at position 6. Then, $W$ is advanced by 4 positions to
position 10. However, since the length of the remaining input text in
$W$, i.e. 21, is smaller than $lsp$, which means there can not be any
pattern starting at that position, the whole scanning procedure is
terminated.

\section{The verification module}
\label{sec:verification}

The verification module is the core of the whole matching engine,
since it determines the worst case performance of the engine. The
verification module in our engine is based on a new data structure
called \emph{Adaptive Matching Tree} (AMT) which is a modification of
the classical \emph{trie} structure. Like the \emph{trie}, but in a
more flexible and space efficient way, the AMT enables a fast
membership query which decides whether a given target string exists in
the pattern set. In the next subsections, we will introduce the
construction of the AMT and some further optimization techniques.

\subsection{AMT Construction}
\label{subsec:amt}

The AMT is constructed from the pattern set, and in general, this
construction includes five major steps:

\begin{enumerate}
\item Create a suffix set $SF$, and put each pattern in the pattern
  set into $SF$, where each pattern is regarded as a suffix of itself.
\item Compute the \emph{length of the shortest suffix} ($lss$) in
  $SF$. For each suffix in $SF$, remove its length-$lss$ prefix, and
  all the removed prefixes form a prefix set $PF$. Discard the
  repetitions in $PF$, and calculate $|PF|$ which refers to the
  \emph{number of distinct prefixes} ($ndp$).
\item Create a tree \emph{node} to hold the prefixes in $PF$. The
  \emph{node} is actually an index, in which each prefix serves as a
  key. The inner structure of the \emph{node} is constructed
  adaptively according to $ndp$ and $lss$ that have been computed in
  step 2.
\item Divide $SF$ by putting together all the suffixes with the same
  prefix removed, to form a sub-suffix-set. Associate each
  sub-suffix-set with the corresponding prefix in the tree
  \emph{node}.
\item For each newly created sub-suffix-set in step 3, repeat the same
  procedure from step 2.
\end{enumerate}

\begin{figure}[htbp]
  \centering
  \includegraphics[width=\textwidth]{./eps/AMT}
  \caption{Constructing the AMT from a pattern set. Each arrow
    in the AMT indicates a child pointer.}
  \label{fig:AMT}
\end{figure}

As an example, Figure \ref{fig:AMT} shows the construction of AMT from
a pattern set $P = \{p_1,\, p_2,\, \dots,\, p_{13}\}$. Note that,
since each pattern is a suffix of itself, the whole pattern set $P$
can be regarded as a suffix set $SF$.  First, since $p_6$ is the
shortest pattern(suffix) in $P$, we have $lss = |p_6| = 2$. Then the
length-2 prefix of each pattern(suffix) is removed, and the removed
prefixes form a prefix set:
$PF = \{p_1^{+2} = aa,\; p_2^{+2} = aa,\; \dots,\; p_{13}^{+2} =
cc\}$.  After discarding the repetitions in $PF$, only 3 distinct
prefixes are left: $PF = \{aa,\, bb,\, cc\}$, and we have
$ndp = |PF| = 3$.

Then the first tree node, which is actually the root of AMT, is
created to hold the prefixes in $PF$. Each prefix serves as a
\emph{key} in the root and has a pointer (initialized to $NULL$) which
will be used to associate with the child node in the future. As will
be seen latter, the structure of root is selected adaptively according
to two parameters: $lss=2$ and $ndp=3$.

After creation of the root, the suffixes left in $P$ are grouped based
on their lost 2-symbol prefixes: those suffixes having the same
prefixes removed are grouped together to form a sub-suffix-set. In
order to record the correspondence between the sub-suffix-set and the
lost prefix, the tuple (\emph{node.key,\; sub-suffix-set}) is used:
the suffixes in the \emph{sub-suffix-set} have the same prefix
\emph{key} removed, where the \emph{key} is held as a key in the newly
created \emph{node}. Using this notation, there are three tuples
created after grouping the suffixes left in $P$:
$tp_1 = (root.aa,\; SF_1=\{p_1^{-2},\, p_2^{-2},\, p_3^{-2},\,
p_4^{-2},\, p_5^{-2},\, p_7^{-2}\})$,\,
$tp_2 = (root.bb,\; SF_2=\{p_8^{-2},\, p_9^{-2},\, p_{10}^{-2}\})$ and
$tp_3 = (root.cc,\; SF_3=\{p_{11}^{-2},\, p_{12}^{-2},\,
p_{13}^{-2}\})$. This indicates that $P$ is divided into three
sub-suffix-sets: $SF_1$, $SF_2$, $SF_3$, and each sub-suffix-set
corresponds to a prefix held as a key in the root (it also means that
the root may have three child nodes). Note that, the shortest
suffix(pattern) $p_6$ has disappeared in the sub-suffix-sets, since
after removing the length-2 prefix of $p_6$, there is nothing left.

The same procedure is repeated on each of the three created
sub-suffix-sets. As a further example, $SF_1$ of $tp_1$ is
processed. First, since $lss = |p_7^{-2}| = 6$, the length-$6$ prefix
of each suffix in $SF_1$ is removed to form a prefix set:
$PF_1 = \{(p_1^{-2})^{+6},\, (p_2^{-2})^{+6},\, (p_3^{-2})^{+6},\,
(p_4^{-2})^{+6},\, (p_5^{-2})^{+6},\, (p_7^{-2})^{+6}\}$. After
discarding the duplicates in $PF_1$, only three distinct prefixes are
left: $PF_1 = \{a^6,\, e^6,\, b^6\}$ (we use the notation $c^n$ to
denote a length-$n$ string consisting of the same character $c$).
Then a new tree node $node_1$ is built adaptively upon $PF_1$ to hold
its prefixes as keys.  According to the first component of $tp_1$
(i.e. $root.aa$), $node_1$ is associated with the key $aa$ in the root
by a child pointer, which makes it to be the first child of root. Once
again, the suffixes left in $SF_1$ are grouped based on their lost
length-6 prefixes, which formed three new tuples:
$tp_4 = (node_1.a^6,\; SF_4=\{p_1^{-8}\})$,
$tp_5 = (node_1.e^6,\; SF_5=\{p_2^{-8},\, p_3^{-8},\, p_4^{-8}\})$ and
$tp_6 = (node_1.b^6,\; SF_6=\{p_5^{-8}\})$.

Note that, in order to mark the end of a pattern, once the last part
of the pattern has been removed and then stored in a node, that last
part is marked with an asterisk(*) in the corresponding node. As a
result of this, any path from the root to a key that marked with an
asterisk represents a complete pattern. It can be seen that, the
patterns are stored implicitly in the AMT and can be reconstructed
from the paths ended with an asterisk.

During the construction of AMT, we use a \emph{breadth-first} strategy
to process the sub-suffix-sets and create tree nodes (which means the
next sub-suffix-set to be processed is $SF_2$ rather than $SF_4$). For
the purpose of tracing the order in which the sub-suffix-sets are
processed, a \emph{first-in-first-out} queue is employed to maintain
the created tuples. The tuple at the beginning of the queue contains
the sub-suffix-set that will be processed next, while the newly
created tuples are inserted to the end of the queue in order. In our
example, the root node is built initially, then $tp_1$, $tp_2$ and
$tp_3$ are inserted to the queue. Next $SF_1$ of $tp_1$ is processed
and the newly created tuples $tp_4$, $tp_5$ and $tp_6$ are inserted to
the queue in order. Subsequently, $SF_2$, $SF_3$, $SF_4$, $SF_5$,
$SF_6$, $\dots$ are processed in sequence. Once there is no tuple left
in the queue, the whole AMT has been constructed.

\begin{algorithm}
  \caption{Constructing the AMT}\scriptsize
  \label{alg:amt}
  \begin{algorithmic}[1]
    \REQUIRE The pattern set $P$
    \ENSURE The corresponding AMT
    \STATE
    \STATE $Q \leftarrow$ Create an empty queue
    \STATE $push\_queue((NULL,\,P),\; Q)$
    \STATE
    \WHILE{$Q$ is not empty}
    \STATE $(parent\_node.key,\; SF) \leftarrow pop\_queue(Q)$
    \STATE $lss \leftarrow$ The length of the shortest pattern in $SF$
    \FOR{each $suf \in SF$}
    \IF{$|suf|=lss$}
    \STATE Mark $suf^{+lss}$ to be the pattern end
    \ENDIF
    \STATE Remove $suf^{+lss}$ of $suf$, and put $suf^{+lss}$ into $PF$
    \ENDFOR
    \STATE Discard the repetitions in $PF$, and let $ndp \leftarrow |PF|$
    \STATE According to $lss$ and $ndp$, adaptively create a
    $new\_node$ to hold the prefixes in
    $PF$
    \IF{$(parent\_node.key = NULL)$}
    \STATE $root \leftarrow new\_node$
    \ELSE
    \STATE Associate $new\_node$
    with $parent\_node.key$ by a child pointer
    \ENDIF
    \STATE $TP \leftarrow \{(new\_node.pf,\, SSF) \mid SSF \subseteq SF\; and
    \ \forall \ p,\,q \in SSF: p,\,q$ have the same length-$lss$ prefix
    $pf$ removed\}\
    \FOR{each $tp \in TP$}
    \STATE $push\_queue(Q,\,tp)$
    \ENDFOR
    \ENDWHILE
    \STATE
    \RETURN $root$.
  \end{algorithmic}
\end{algorithm}

The framework of constructing the AMT from a pattern set is
illustrated in Algorithm \ref{alg:amt}. Firstly, an empty queue $Q$ is
created in line 2 for maintaining the tuples. Then the tuple
$(NULL, P)$ is inserted to $Q$ in line 3 by function $push\_queue$
which always inserts an element to the end of the queue. Since $P$ is
the initial pattern set here, the first component of the tuple is
$NULL$, which means there is no prefix removed from $P$ yet. By this
way, the creation of the root can be combined with the creation of
other tree nodes in the following \textbf{while} loop.

If $Q$ is not empty, the \textbf{while} loop body from line 5 to line
25 constructs a tree node based on the first tuple in $Q$. The tuple
at the beginning of $Q$ is fetched by the $pop\_queue(Q)$ function in
line 6: the suffix set to be processed is assigned to $SF$, and the
corresponding prefix of $SF$ in the parent node is denoted by
$parent\_node.key$. Then the $lss$ of $SF$ is computed in line 7. In
the inner \textbf{for} loop from line 8 to line 13, the length-$lss$
prefix of each suffix in $SF$ is removed, and the prefixes are
collected to form the prefix set $PF$. If a prefix is the last part of
some pattern, it is marked as a pattern end. Line 14 discards the
repetitions in $PF$, after that, a new tree node is created adaptively
to hold the prefixes $PF$ in line 15. The \textbf{if} statement in
line 16 decides whether the newly created node is the root node or
not: if the prefix component of the current tuple is $NULL$, the new
node is the root node as stated before; otherwise, the new node is a
child node which will be then associated with the corresponding prefix
in its parent node. Next in line 21, the suffixes left in $SF$ are
grouped based on their lost prefixes, and each group and the
corresponding prefix forms a tuple. Finally, the created tuples are
inserted orderly to the end of $Q$ by the $push\_queue$ function. Once
there is no tuple left in $Q$, the whole AMT has been completely
constructed, and the root of AMT is returned.

\subsection{Verification}
\label{subsec:matching}

As stated in the filter module, once a suspicious position in the text
has been detected, the verification module checks that position for
real pattern occurrences. Specifically, for such a suspicious position
$i$, the engine searches a sequence of substrings of the text, where
the first substring starts at $i$, in the corresponding nodes of AMT
from the root. If any substring successfully matched a key that is
marked as a pattern end in some node, a pattern is declared to be
found at position $i$. However, if there is a mismatch or the
searching goes beyond the leaves of AMT, the engine immediately
forwards to the next position $i+1$ and invokes the filter again.

The pseudocode of the verification procedure is shown in Algorithm
\ref{alg:matching}. Suppose $i$ is a suspicious position of the text
detected by the filter module. The variable $m\_len$ is the total
length of the sub-strings of the text who have successfully matched
with keys in AMT nodes. The variable $node$ points to the current
matching node in the AMT. In addition, since all the keys in a node
have the same length, the notation $key\_len(node)$ is used to denote
the length of the keys in $node$.

\begin{algorithm}
  \caption{Verification}\scriptsize
  \label{alg:matching}
  \begin{algorithmic}[1]
    \REQUIRE ~~\\
    A suspicious position $i$ in the text $T$\\
    The AMT built from the pattern set $P$\\
    \ENSURE ~~\\
    The patterns (if any) start at $i$
    \STATE
    \STATE $m\_len \leftarrow 0$
    \STATE $node \leftarrow root$ of AMT
    \STATE
    \WHILE{$node \neq NULL$ and $\exists key \in node: key =
      T[i+m\_len, \, key\_len(node)]$}
    \STATE $m\_len \leftarrow m\_len + key\_len(node)$
    \IF{$key$ is marked as a pattern end}
    \STATE Report that the pattern $T[i,\,m\_len]$ is found at position $i$
    \ENDIF
    \STATE $node \leftarrow$ The child of $node$ corresponding to $node.key$
    \ENDWHILE
  \end{algorithmic}
\end{algorithm}

For the suspicious position $i$, the \textbf{while} loop from line 5
to line 11 checks whether there are patterns starting at $i$. The
checking starts form the root of AMT: if the corresponding sub-string
$T[i+m\_len, \, key\_len(node)]$ matches some $key$ in the current
$node$, the totally matched length $m\_len$ is increased by the length
of the key. At the same time, if the matched $key$ is also marked as a
pattern end, the verification module will report that a pattern,
i.e. $T[i,\,m\_len]$, is found at position $i$. Then the checking
transfers to the child of $node.key$ and makes that child node to be
the current node. Once there is a mismatch or the current node goes
beyond the leaves of AMT (i.e.  $node = NULL$), the checking
terminates, and the engine forwards to the next position $i+1$.

Note that, since the nodes in AMT have various types of inner
structures, the search of the target string in a node, must use the
search routine specified to the type of that node.

\begin{figure}[htbp]
  \centering
  \includegraphics[width=0.75\textwidth]{./eps/match}
  \caption{The verification procedure for the position $i$ in the
    text.}
  \label{fig:matching}
\end{figure}

Next, an example is given in figure \ref{fig:matching} to illuminate
the verification for a suspicious position $i$ of a given text string
$T$, using the AMT built in section \ref{subsec:amt}. The verification
starts from the root of AMT: according to $key\_len(root)=2$, the same
length sub-string starting at position $i$ of $T$, which is
$T[i,\,2]=aa$, is taken to check whether it's one of the keys in
root. The result is: $aa \in root$, and $aa$ is also marked as a
pattern end. Therefore the pattern $T[i,\,2]=aa$ ($p_6$ in $P$) is
found at position $i$. Then the checking transfers to $node_1$, which
is the child node of $root.aa$.

At $node_1$, since $key\_len(node_1)=6$, the same length sub-string
$T[i+2,\,6]=e^6$, which follows $T[i,\,2]$, is trying to match with
some key in $node_1$. The result is that: $e^6 \in node_1$ but it is
not a pattern end. Then the checking simply transfers to $node_5$,
which is the child of $node_1.e^6$.

At $node_5$, since the next sub-string $T[i+8,\,5]=u^5 \in node_5$ and
$u^5$ is marked as a pattern end, another pattern
$T[i,\,13]=a^2e^6u^5$ ($p_4$ in $P$) is found at position $i$. Then
the checking goes to the child of $node_5.u^5$, i.e.  $node_{10}$.

At $node_{10}$, since $T[i+13,\,4]=sscc \notin node_{10}$, the
verification for position $i$ terminates. And next, the engine will
forward to position $i+1$ and invoke the filter module for that
position.

From above it can be seen that, each verification procedure actually
corresponds to a \emph{verification path} in the AMT, whose tree nodes
have been compared with the text in the former process. In particular,
the verification path of the given example is:
$root \rightarrow node_1 \rightarrow node_5 \rightarrow node_{10}$,
witch is indicated by the dashed arrows in Figure \ref{fig:matching}.


\subsection{Adaptive creation of the tree nodes}
\label{subsec:nodes}

As stated in section \ref{subsec:amt}, each node of AMT has a specific
index structure to hold the prefixes of a prefix set as the keys. The
type of the index structure is chosen adaptively according to the
features of the prefix set. Since all the keys in the prefix set have
the same length, the features of the prefix set can be mainly
characterized by two values: the \emph{length} and \emph{number} of
the keys in the prefix set. These two values have been respectively
symbolized by $lss$ and $ndp$ in Section \ref{subsec:amt}.

In this work, three different kinds of structures,
i.e. \emph{character map}, \emph{string array} and \emph{hash table}
can be adopted for prefix sets with different $lss$ and $ndp$
combinations. Next, the detailed descriptions of these structures will
be presented.

\subsubsection{Character map}

Given a prefix set, if the length of prefixes (keys) is equal to 1 ($lss=1$),
i.e. each key in the prefix set is just a single character, the
space-efficient \emph{character map} structure, which was proposed by
Leis \cite{Leis2013}, will be adopted as the tree node. There are four
sub-types of character maps with different capacities for different
number of keys ($ndp$), where $1 \leq ndp \leq 256$.

\begin{figure}[htbp]
  \centering
  \includegraphics[width=0.75\textwidth]{./eps/Maps}
  \caption{The four sub-types of character maps. The triangles
    represent the corresponding child nodes of the keys.}
  \label{fig:character map}
\end{figure}

Figure \ref{fig:character map} illustrates the four sub-types of
character maps which are named according to their maximum
capacity. Instead of using one array of (\emph{key, child pointer})
pairs, the key part and the child pointer part are stored in the
separate arrays, which is able to keep the node structure compact
while supporting efficient search.

\begin{itemize}
\item \textbf{Map 4} (for $1 \leq ndp \leq 4$): The smallest map type
  can store up to 4 keys. It uses an array of 4 entries for keys and
  another array of the same length for child pointers. The keys and
  child pointers are stored at corresponding positions in their
  arrays, and the keys are sorted according to their ACSII
  values. Once the target character is found in the key array, its
  child pointer can be located through the same position in the
  pointer array. Figure \ref{fig:character map} (a) shows a Map 4
  structure with three keys: $f$, $n$ and $p$, where the triangles
  with the keys inside represent the corresponding child nodes.

\item \textbf{Map 16} (for $5 \leq ndp \leq 16$): This map type is
  used for storing 5 to 16 keys. It has the similar structure as Map
  4, but both arrays have up to 16 entries. A target character can be
  retrieved efficiently by a binary search in the key array. Figure
  \ref{fig:character map} (b) shows a Map 16 structure with 14 keys.


\item \textbf{Map 48} (for $17 \leq ndp \leq 48$): As the number of
  keys increases, searching in the key array becomes
  expensive. Therefore, maps with more than 16 (but less than 49) keys
  do not store the keys explicitly. Instead, a 256-element \emph{index
    array} is used, which can be directly indexed by the ASCII value
  of the target character. This array stores only the array indexes
  (small inters in the range of $[0,47]$) of another pointer array
  that contains up to 48 child pointers. In this way, the storage
  space can be saved comparing with storing pointers directly, because
  each array index only requires one byte. Figure \ref{fig:character
    map} (c) shows a Map 48 structure, where the ASCII values of $a$,
  $b$ and $c$ are 97, 98 and 99 respectively.

\item \textbf{Map 256} (for $49 \leq ndp \leq 256$): The largest map
  type is simply an array of 256 child pointers with each pointer
  initialized to $NULL$. It is used for storing 49 to 256 keys.  In
  this kind of character map, the child node can be found directly
  through the array index which is the ASCII value of the target
  character. Different from other character maps, in Map 256, there is
  only one array, and it is not necessary to carry out the additional
  indirect access. Therefore, if most entries are not empty, this
  representation is also very space efficient. Figure
  \ref{fig:character map} (d) shows a Map 256 structure, where only
  the pointer array is adopted.
\end{itemize}

The pseudocode of searching in a node, whose type is character map, is
given in Algorithm \ref{alg:character map}.

\begin{algorithm}
  \caption{Searching in a node whose type is character map}\scriptsize
  \label{alg:character map}
  \begin{algorithmic}[1]
    \REQUIRE ~~\\
    A tree $node$ whose type is character map\\
    The target character $t\_ch$.
    \ENSURE ~~\\
    The child node of $node.t\_ch$ (possibly $NULL$).
    \STATE
    \STATE $count \leftarrow$ Number of keys in the character map
    \STATE
    \SWITCH{The type of the character map}
    \CASE{\textsf{Map 4}}
    \FOR{$i \leftarrow 0$ to $count-1$}
    \IF{$keys[i]=t\_ch$}
    \RETURN $child\_pointers[i]$
    \ENDIF
    \ENDFOR
    \STATE Report a mismatch.
    \ENDCASE
    \STATE
    \CASE{\textsf{Map 16}}
    \STATE $low \leftarrow 0, high \leftarrow count-1$
    \WHILE{$low \le high$}
    \STATE $mid \leftarrow \lfloor (low+high)/2 \rfloor$
    \IF{$t\_ch=keys[mid]$}
    \RETURN $child\_pointers[mid]$
    \ELSIF{$t\_ch<keys[mid]$}
    \STATE $high \leftarrow mid-1$
    \ELSE
    \STATE $low \leftarrow mid+1$
    \ENDIF
    \ENDWHILE
    \STATE Report a mismatch.
    \ENDCASE
    \STATE
    \CASE{\textsf{Map 48}}
    \IF{$index[t\_ch] \neq NULL$}
    \RETURN $child\_pointers[index[t\_ch]]$
    \ELSE
    \STATE Report a mismatch.
    \ENDIF
    \ENDCASE
    \STATE
    \CASE{\textsf{Map 256}}
    \RETURN $child\_pointers[t\_ch]$
    \ENDCASE
    \ENDPWITCH
  \end{algorithmic}
\end{algorithm}

\subsubsection{String array}
\label{sec:string array}

For the prefix set whose length of keys is greater than 1 ($lss > 1$)
and the number of keys is not greater than 100 ($ndp \leq 100$), the
\emph{string array} structure is adopted as a tree node. Similar to
the Map 4 and Map 16 structures, the keys are stored in
lexicographical order in a separate key array with $ndp \times lss$
bytes, in which each key takes $lss$ bytes. The child pointers are
stored at the corresponding positions in another pointer array. Figure
\ref{fig:string array} illustrates a string array structure holding
three keys: $aaa$, $bbb$ and $ccc$.

\begin{figure}[htbp]
  \centering
  \includegraphics[width=0.45\textwidth]{./eps/string_array}
  \caption{The string array structure with three keys: $aaa,\, bbb,\,
    ccc$.}
  \label{fig:string array}
\end{figure}

For efficiency, if the number of keys is less than 5, a naive linear
search is used to search the target string in the key array;
otherwise, a more efficient (but complicated) binary search algorithm
is adopted.  The pseudocode of the searching in a node whose type is
string array is depicted in Algorithm \ref{alg:string array}, where we
use the notation $A \prec B$ to denote that string $A$ is
\emph{lexicographically smaller} than string $B$. For the string array
whose number of keys is less than 5, the target string is compared
sequentially with the keys in the key array. If the target string
matched some key, return the child pointer at the corresponding
position in the pointer array; otherwise, report a mismatch. On the
other hand, if the number of keys is greater than 4, the target string
is searched by a binary search routine from the middle element of the
key array.

\begin{algorithm}
  \caption{Searching in a node whose type is string array}\scriptsize
  \label{alg:string array}
  \begin{algorithmic}[1]
    \REQUIRE ~~\\
    A tree $node$ whose type is string array; \\
    The target string $t\_str$.\\
    \ENSURE ~~\\
    The child node of $node.t\_str$ (possibly $NULL$).\\
    \STATE
    \STATE $count \leftarrow$ The number of keys in $node$
    \STATE
    \IF{$count < 5$}
    \STATE $i \leftarrow 0$
    \WHILE{$i < count$ and $keys[i] \prec t\_str$}
    \STATE $i \leftarrow i+1$
    \ENDWHILE
    \IF{$i < count$ and $keys[i]=t\_str$}
    \RETURN $child\_pointers[i]$
    \ELSE
    \STATE Report a mismatch
    \ENDIF
    \ELSE
    \STATE $low \leftarrow 0$, $high \leftarrow count-1$
    \WHILE{$low \leq high$}
    \STATE $mid \leftarrow \lfloor (low+high)/2 \rfloor$
    \IF{$t\_str = keys[mid]$}
    \RETURN $child\_pointers[mid]$
    \ELSIF{$t\_str < keys[mid]$}
    \STATE $high \leftarrow mid - 1$
    \ELSE
    \STATE $low \leftarrow mid + 1$
    \ENDIF
    \ENDWHILE
    \STATE Report a mismatch.
    \ENDIF
  \end{algorithmic}
\end{algorithm}

\subsubsection{Hash table}
\label{sec:hash table}

For the prefix set whose number of keys is greater than 100 ($lss > 1$
and $ndp > 100$), searching in the string array structure becomes
inefficient even using a binary search. In this case, a more fast (but
heavy) data structure --- \emph{hash table} is adopted to deal with
large number of keys. It is worth noting that, the construction of the
hash table is directly based on the suffix set from which the prefix
set is formed rather than the prefix set which is the basis of
building other node structures. In our implementation, a hash table is
an array of child pointers with each pointer initialized to $NULL$,
and we also utilize a \emph{string hash function} which transforms a
string to a positive integer.

% Indeed, when computing $ndp$, we
% merely count rather than really constructing the prefix set and then
% computing its $ndp$

In particular, given a suffix set $SF$, the size of the hash table
(denoted by $table\_size$) is determined by $ndp$ and a given load
factor $lf$ (ratio of $ndp$ to $table\_size$), i.e.
$table\_size = \lceil ndp\,/\,lf \rceil$. For example, with the $ndp$
of 1000 and a load factor of $70\%$, the $table\_size$ is
$\lceil 1000/0.7 \rceil = 1429$. Note that, the $ndp$ here is defined
to be the number of distinct length-$lss$ prefixes of the suffixes in
$SF$, which is equal to the size of the prefix set derived from $SF$
(after removing the repetitions).

Algorithm \ref{alg:hash} shows the pseudocode of building a hash table
based on $SF$. $SF$ is firstly partitioned into small suffix sets by
hashing: for each suffix $suf \in SF$, its prefix $suf^{+lss}$ is
hashed to an integer $i$ between 0 and $table\_size-1$ by the string
hash function, then $suf$ is associated with the $i$-th slot of the
hash table. After that, the suffixes whose length-$lss$ prefixes
hashing to the same value are associated with the same slot of the
hash table. Then, to address the hash collisions, for each slot that
is not $NULL$, the associated suffixes form a small suffix set $SSF$,
and a new tree node is created adaptively based on the prefix set
derived from $SSF$, as we have shown in the \textbf{while} loop of
Algorithm \ref{alg:amt}. The newly created tree node is then
associated with the corresponding slot of the hash table. As a result,
many tree nodes with various types can be associated within one hash
table.

% The suffixes whose length-$lss$ prefixes hashed to the same
% value $i$ are collected together associated with the $i$-th slot and
% forms a small suffix set. (Note that, we do not remove the prefixes of
% the suffixes when partitioning $SF$.) Then, to address the collisions,
% for each non-$NULL$ slot of the hash table, a tree node is built based
% on the small suffix set associated with that slot as stated
% before. Hence, many tree nodes with various structures can be
% incorporated in one hash table.


\begin{algorithm}
  \caption{Building a hash table}\scriptsize
  \label{alg:hash}
  \begin{algorithmic}[1]
    \REQUIRE ~~\\
    The suffix set $SF$ and the load factor $lf$. \\
    \ENSURE ~~\\
    The hash table.\\
    \STATE
    \STATE $ndp \leftarrow$ The number of distinct length-$lss$
    prefixes of suffixes in $SF$
    \STATE $table\_size \leftarrow \lceil ndp\,/\,lf \rceil$
    \STATE $hash\_table \leftarrow$ Create an array  of
    $table\_size$ pointers with each pointer initialized to $NULL$
    \STATE
    \FOR{each $suf \in SF$}
    \STATE $i \leftarrow Hash(suf^{+lss})$
    \STATE Associate $suf$ with $hash\_table[i]$
    \ENDFOR
    \STATE
    \FOR{$i \leftarrow 0$ to $table\_size - 1$}
    \IF{$hash\_table[i]\, \neq\, NULL$}
    \STATE $SSF \leftarrow \{suf\,|\,suf\in SF\; and\; Hash(suf^{+lss})=i\}$
    \STATE $new\_node \leftarrow$ Adaptively create a tree node from
    $SSF$, as shown in the \textbf{while} loop of Algorithm \ref{alg:amt}
    \STATE Associate $new\_node$ with $hash\_table[i]$
    \ENDIF
    \ENDFOR
    \STATE
    \RETURN $hash\_table$
  \end{algorithmic}
\end{algorithm}

Given a target string of length $lss$, the hash value of that string
is computed and used as the index of the hash table. If the
corresponding slot is $NULL$, which means there is a mismatch, the
verification terminates and the engine moves to the next position;
otherwise, the process goes to the tree node associated with that slot
and compares the node with the corresponding substring in the text.

The string hash function adopted can have a significant effect on the
performance of verifying. In our implementation, for each matching
task, the hash function is selected randomly from a \emph{hash
  function family} $H$. The hash function family we adopted is the
fast \emph{shift-add-xor} hash family proposed by Ramakrishna
\cite{Ramakrishna1997}, which is claimed to be both uniform and
universal.

\subsection{Further Improvements}
\label{sec:further improments}

The verification module presented above is pretty efficient, but there
are still some techniques that can further improve the time and space
efficiency of this module.

\subsubsection{Dividing the AMT}
\label{sec:divide amt}

The verification module we have just built is based on a big single
AMT which is constructed from the whole pattern set. In fact, we can
divide the big AMT into small ones and enable each verifying procedure
to refer to only one small AMT, by which the time for verifying can be
reduced. The division is based on the following observations.

As stated in Section \ref{sec:filter}, the filtering is based on the
last $k$-symbol block of the substring in the sliding window. In
particular, for a position $i$ of the text $T$, if some pattern $p_j$
starts at $i$, there must be
$Hash[T[i+lss-k,k]] = Hash[p_j[lss-k+1,k]]$ (we call
$Hash[p_j[lss-k+1,k]]$ the \emph{signature} of $p_j$). This means that
we can group the patterns according to their signatures: patterns with
the same signature are grouped and form a sub-pattern-set, and for
each sub-pattern-set we build a (small) AMT. In addition, an index
table is adopted to allow the corresponding AMT be retrieved in $O(1)$
time by its signature. Now, the verification module consists of an
index table as well as several small AMTs.

In the filtering phase, once a suspicious position $i$ in $T$ is
detected, the value $Hash[T[i+lss-k,k]]$ will be used as the index of
the index table to find the corresponding small AMT for verifying. As
the scale of the AMT shrunk, the time for each verifying is reduced
accordingly.

\subsubsection{Node Merging}
\label{sec:node merge}

Similar to the ``path compression'' technique that is widely used in
many trie-like structures, the nodes in a path of AMT, each of which
has only one key, can be merged into one node. This technique can
improve both time and space efficiency of AMT. To illustrate its
benefits, consider the pattern set
$P=\{p_1=e^6u^5sscc,\; p_2=e^6u^5,\; p_3=e^6\}$ and the corresponding
AMT constructed by Algorithm 1. As shown in Fig \ref{fig:merge}, the
built AMT has three nodes, and each contains only one key. In a
typical x86\_64 architecture, this implementation of AMT takes roughly
55 bytes of memory: 15 bytes for the three keys, 16 bytes for the two
child pointers and 24 bytes for the three pointers to the matching
functions. However, if we combine the three keys into one string,
these three nodes can be merged into one node, as a result, the child
pointers are no longer needed and the number of function pointers is
reduced from three to one. The merged node is another type of tree
node in addition to the node types we have introduced in Section
\ref{subsec:nodes}.

\begin{figure}[htbp]
  \centering
  \includegraphics[width=0.6\textwidth]{./eps/node_merge}
  \caption{Node merging.}
  \label{fig:merge}
\end{figure}

For the merged node, the only problem is how to distinguish among the
keys after they are concatenated. To address this issue, a
\emph{length array} is used to store the length of each key, where
each array element takes only one byte (using the \emph{unsigned char}
type). Through the length of each key, we can delimit each original
key from others, and thus, searching the target string in the merged
node can be trivial.

In general, after node merging, the AMT contains just one merged node
which only takes 26 bytes (15 bytes for the keys, 3 bytes for the
length array, and 8 bytes for the matching function pointer), thus, 29
bytes are saved comparing with the original AMT. In addition, only one
random access to the memory is needed to get the merged node (the
original AMT needs three random memory accesses to get all its nodes),
thus, the cache miss is reduced and the performance is improved.


\subsubsection{Node splitting}
\label{sec:node split}

In order to further improve the searching efficiency of the tree
nodes, once the keys in a node share a common prefix, that prefix
should best be removed from the keys and form a new single key node
. (Note that, since the prefix is either a character or a string, the
type of the new created node is either Map 4 or String Array.)  This
action can reduce a lot of redundant comparisons while searching a
target key in a node. For instance -- as shown in Fig \ref{fig:split},
consider searching a target key $a^5u$ in a node of String Array type,
where the four keys of the node share a common prefix $a^5$. According
to Algorithm \ref{alg:string array}, the target key have to be
compared sequentially with the four keys, which will totally takes
$6 \times 4 = 24$ byte comparisons to determine the inexistence of
target key. However, if the common prefix $a^5$ is removed out of the
keys and form another new node as shown in Fig \ref{fig:split}, for
the same target key, only 9 byte comparisons are needed for verifying
(5 comparisons for $node_1$ and another 4 comparisons for
$node_2$). Obviously, comparing with only one node, the two nodes
version can reduce 15 comparisons.

\begin{figure}[htbp]
  \centering
  \includegraphics[width=0.4\textwidth]{./eps/node_split}
  \caption{Node splitting.}
  \label{fig:split}
\end{figure}



\section{Experimental Results}
\label{sec:experiments}

In this section, we evaluate the performance of the proposed fast
engine for multi-pattern matching (\textsf{FEMPM} for short), and
compare it with other four algorithms: the \textsf{AC} and \textsf{WM}
algorithms, which are two classic ``baseline'' algorithms for MPM; the
\textsf{MASM} and \textsf{Pre-filter+AC} algorithms, which are two
state-of-the art MPM algorithms with good efficiency in practically
matching. All these algorithms are evaluated in two aspects: the
robustness of these algorithms under pattern sets with various $lsp$s
and the scalability of the algorithms under pattern sets with various
number of patterns.

\subsection{Simulation Settings}


The experiments are conducted on a PC with an Intel core i7 2.93GHz
CPU, 8GB of RAM and 1TB Disk Driver; the operating system is Windows 7
Professional (64-bits). All the testing algorithms are implemented in
C/C++ within the IDE of \emph{Code::Blocks}. The test data used in the
experiments is the real-world English text from the Pizza\;\&\;Chili
corpus \cite{pizzachil}. The patterns are extracted randomly from the
text to form the pattern sets.

The parameter configuration in \textsf{FEMPM} in our experiment is
given as follows. For the String Arrays, if the number of strings
exceeds \textbf{4}, the binary search is used to search the target
string in the key array; otherwise, the linear search is adopted. For
the Hash Tables, the load factor is set to \textbf{0.5}; the
\emph{seed} of the hash function is generated randomly in the range of
$\mathbf{1} \sim \mathbf{50}$; the shift values \emph{L} and \emph{R}
are set to \textbf{2} and \textbf{6} respectively. Once the $ndp$ is
lager than \textbf{100}, the Hash Table is built instead of the String
Array. These choices of parameters have been extensively tested on a
wide range of pattern sets, and are considered to be very efficient
for a typical match.

\subsection{Robustness Evaluation}

As we've mentioned before, many MPM algorithms are sensitive to the
length of the shortest pattern ($lsp$) in the pattern set. So first,
it is necessary to measure the robustness of \textsf{FEMPM} and other
investigated algorithms under pattern sets with various $lsp$s. In
this evaluation, there are totally 9 pattern sets for testing with
$lsp$s ranging from 2 to 10, and each pattern set has a fixed number
of $10^5$ patterns. The size of text string is fixed to 200 MB. The
characteristics of the testing pattern sets are shown in Table
\ref{tab:lsps}, which includes: the length of the shortest pattern
(LSP), the length of the longest pattern (LLP), the average length of
the patterns (ALP), the standard deviation of the pattern lengths
(SD), the total length of the patterns (TLP) and the number of
patterns in the pattern set (Count).

\begin{table}
  \centering
  \caption{The characteristics of pattern sets with various LSPs.}
  \scriptsize
  \label{tab:lsps}
  \begin{tabular}{rrrrrrr}
    \hline
    Pattern Set & LSP  & LLP  & ALP & SD & TLP & Count\\
    \hline
    $P_1$ & 2 & 50 & 25.9 & 14.1 & 2,599,020 & $10^5$\\
    $P_2$ & 3 & 50 & 26.5 & 13.9 & 2,646,762 & $10^5$\\
    $P_3$ & 4 & 50 & 27.0 & 13.6 & 2,704,091 & $10^5$\\
    $P_4$ & 5 & 50 & 27.6 & 13.2 & 2,745,545 & $10^5$\\
    $P_5$ & 6 & 50 & 27.9 & 13.0 & 2,793,178 & $10^5$\\
    $P_6$ & 7 & 50 & 28.4 & 12.7 & 2,843,580 & $10^5$\\
    $P_7$ & 8 & 50 & 29.0 & 12.4 & 2,903,343 & $10^5$\\
    $P_8$ & 9 & 50 & 29.5 & 12.1 & 2,948,372 & $10^5$\\
    $P_9$ &10 & 50 & 30.0 & 11.8 & 2,998,992 & $10^5$\\
    \hline
  \end{tabular}
\end{table}

\begin{figure}[htbp]
  \centering
  \includegraphics[width=0.6\textwidth]{./eps/lsp}
  \caption{Matching times for various lsps.}
  \label{fig:lsp}
\end{figure}

Figure \ref{fig:lsp} presents the mean matching time (in seconds) used
by these algorithms based on 5 independent runs. In general, the
\textsf{AC} algorithm is not affected much by the changes in
$lsp$. However, the DFA built by this algorithm consumes a huge amount
of memory. What's worse is that, the frequent and unpredictable
state/failure transitions will lead to a lot of cache misses during
matching, and these result in the low performance of \textsf{AC}. On
the other side, it can be seen from Figure \ref{fig:lsp} that the
\textsf{WM} algorithm is affected much by the changes in $lsp$: it
outperforms \textsf{AC} on the pattern sets whose $lsp$ is larger than
2, but for the pattern sets with small $lsp$, the performance of
\textsf{WM} degrades dramatically. This is due to the ``skip
strategy'' of \textsf{WM} algorithm that does not work well for the
pattern sets with small $lsp$s, which means most of the positions in
the text need to be checked by the HASH table of \textsf{WM}, and this
takes a lot of time. The \textsf{Pre-filter+AC} algorithm outperforms
all other algorithms except for \textsf{FEMPM} for pattern sets whose
$lsp$ is larger than 4. Similarly, this algorithm also uses a skip
strategy: once a position of the text is filtered out by the
Pre-filter, at most the following $lsp-1$ positions can be skipped
over. Consequently, for pattern sets with small $lsp$, this algorithm
can not skip much in the text, which will seriously affect its
performance. In contrast, the \textsf{MASM} algorithm performs well
for small $lsp$ ($lsp \leq 4$). This algorithm first compacts the
pattern set by a prefix tree: if a pattern $A$ is a prefix of another
pattern $B$, $A$ can be merged into $B$. Then the algorithm builds a
binary search tree for the compacted pattern set based on
lexicographical order. Comparing with the long patterns, the short
ones are more likely to be prefixes of others, and thus, more likely
to be eliminated by compacting. Thus, pattern sets with small $lsp$
are well compressed, which will improve both time and space efficiency
of \textsf{MASM} in the matching phase.

From Figure \ref{fig:lsp} we can also see that, \textsf{FEMPM}
outperforms all other investigated algorithms on every
$lsp$. Comparing with \textsf{AC}, the matching time is reduced by
$85\%$ in average for all $lsp$s; even comparing with the fast
\textsf{Pre-filter+AC} algorithm, the matching time is reduced by
$10\% \sim 20\%$ for medium $lsp$s ($lsp \geq 4$) and by $70\%$ for
small $lsp$s ($lsp < 4$). Although the filter part of \textsf{FEMPM},
which likes that of \textsf{Pre-filter+AC}, is not suitable for small
$lsp$, the verification part, i.e. AMT, is more efficient than that of
the \textsf{Pre-filter+AC}. This is mainly due to the adaptivity of
the nodes of AMT, which enables the tree nodes as efficient and
compact as possible. Therefore, no matter what the $lsp$s of the
pattern sets are, the corresponding AMT is able to keep high search
efficiency as well as low memory requirement in the verification
phase. Owing to this, the \textsf{FEMPM} has the best robustness for
all of the testing pattern sets. The statistical data about node types
of corresponding AMTs are shown in table \ref{tab:node types}. The
Single Char and Single String types in the table are actually the Map
4 and String Array types which contain only one element,
respectively. Obviously, as the $lsp$ of the pattern set changes, the
number of nodes of the each type in the AMT changes accordingly, which
reflects the adaptivity of AMT.

Moreover, the performance of \textsf{FEMPM} is very stable and
reliable. In particular, once the length of the text and number of
patterns are fixed, the matching time changes little for various
$lsp$s. As shown in Figure \ref{fig:lsp}, the matching time of
\textsf{FEMPM} keeps almost 10 seconds unchanged for a text of about
200MB and a pattern set with about $10^5$ patterns; for $lsp$s ranging
from $2$ to $10$, the change in the matching time is less than 2
seconds. In fact, as the $lsp$ increases, the average length of the
paths from the root to the leaves in AMT will increase
slightly. Accordingly, in the matching phase, the matching paths for
some text positions may get a little bit longer, which might slightly
affect the matching speed.

\begin{table}[!htp]
  \centering
  \caption{The statistics about the AMTs under various lsps}
  \scriptsize
  \label{tab:node types}
  \begin{tabular}{rrrrrrrrrr}
 \hline
 Lsp &
 Single Char &
 Map 4 &
 Map 16 &
 Map 48 &
 Map 256 &
 Single String &
 String Array   &
 Hash Table &
 Total\\
 \hline
 2  & 2,464 & 2,096 & 663 & 133 & 0 & 63,274 &  21,627 & 10 & 90,267\\
 3  & 2,379 & 1,636 & 385 & 90  & 0 & 64,701 &  22,551 & 10 & 91,752\\
 4  & 2,273 & 1,329 & 201 & 57  & 0 & 66,162 &  22,736 &  1 & 92,759\\
 5  & 2,045 & 1,104 & 112 & 45  & 0 & 68,152 &  22,048 &  1 & 93,507\\
 6  & 1,758 &   899 &  52 & 34  & 0 & 70,179 &  20,818 &  1 & 93,741\\
 7  & 1,560 &   807 &  29 & 33  & 0 & 72,284 &  19,280 &  1 & 93,994\\
 8  & 1,515 &   803 &  20 & 31  & 0 & 73,447 &  18,274 &  1 & 94,091\\
 9  & 1,441 &   788 &  24 & 30  & 0 & 74,584 &  17,120 &  1 & 93,988\\
10  & 1,361 &   794 &  24 & 29  & 0 & 75,094 &  16,427 &  1 & 93,730\\
\hline
  \end{tabular}
\end{table}

\subsection{Scalability Evaluation}

In this section, we evaluate the scalability of the investigated
algorithms by testing them under pattern sets with various numbers of
patterns. As before, the size of the text is fixed to 200MB, and there
are two groups of pattern sets for testing --- the small group and the
large group. The small group contains 9 pattern sets, with the sizes
(number of patterns) increasing from $1 \times 10^5$ to
$9 \times 10^5$ with step size of $10^5$, while the large group
contains 10 pattern sets, with the sizes increasing from $10^6$ to
$10^7$ with step size of $10^6$. The range of the pattern length of
each pattern set is fixed to $5 \sim 50$.

\begin{figure}[htbp]
  \centering
  \includegraphics[width=0.6\textwidth]{./eps/small_group}
  \caption{Matching times for the small group.}
  \label{fig:small_group}
\end{figure}

For the small group, the mean matching times of the investigated
algorithms under 5 independent runs are shown in Figure
\ref{fig:small_group}. The experimental results indicate that, the
performance of the \textsf{AC} algorithm is not good but relative
stable with the number of patterns growing. Also it can not deal with
pattern sets whose size is larger than $7 \times 10^5$ in this testing
, due to out of memory. Although the \textsf{WM} algorithm performs
better than \textsf{AC}, its performance is not stable. For instance,
the matching time only rises about $1s$ as the number of patterns
increases from $2 \times 10^5$ to $3 \times 10^5$, but when the number
of patterns increases from $8 \times 10^5$ to $9 \times 10^5$, the
matching time rises up to $9s$. Thus, we can not estimate the matching
time of \textsf{WM} based on the size of the pattern set. The
\textsf{Pre-filter+AC} performs well as the number of patterns grows,
but its matching time is also unpredictable, e.g. the matching times
are nearly the same as the number of patterns increases from
$4 \times 10^5$ to $5 \times 10^5$. The reason is that: the effect of
the filters used by these two algorithms highly depends on the
contents of the text and patterns themselves. For some pattern sets
whose patterns occurs frequently in the text, the effect of the
filters degrades. On the other hand, as the number of patterns grows,
the matching time of \textsf{MASM} grows more slowly than that of the
\textsf{Pre-filter+AC}, which is due to that the matching time of
\textsf{MASM} mainly relies on the depth of the pipelinded binary
search tree and that depth grows very slow as the number of patterns
increases.

\begin{table}[!htp]
  \scriptsize
  \caption{The statistics about the AMTs for the small group}
  \label{tab:small}
  \begin{tabular}{rrrrrrrrrr}
 \hline
 Size &
 Single Char &
 Map 4 &
 Map 16 &
 Map 48 &
 Map 256 &
 Single String &
 String Array   &
 Hash Table &
 Total\\
\hline
$1 \times 10^5$  &  2,000 &  1,006 &    72 &  18 & 0 &  68,268 &  21,971 & 103 &  93,438 \\
$2 \times 10^5$ &  4,628 &  2,808 &   307 &  42 & 0 & 130,440 &  46,514 & 244 & 184,983 \\
$3 \times 10^5$ &  7,414 &  4,858 &   574 &  84 & 0 & 190,653 &  71,757 & 403 & 275,746 \\
$4 \times 10^5$ & 10,555 &  7,282 &   946 & 139 & 0 & 247,656 &  97,967 & 544 & 365,089 \\
$5 \times 10^5$ & 13,703 & 10,651 & 1,503 & 238 & 4 & 302,350 & 124,528 &  17 & 452,994 \\
$6 \times 10^5$ & 17,335 & 13,536 & 1,954 & 298 & 0 & 356,644 & 151,435 &  31 & 531,233 \\
$7 \times 10^5$ & 21,080 & 16,872 & 2,405 & 388 & 4 & 408,165 & 179,621 &  35 & 628,570 \\
$8 \times 10^5$ & 24,357 & 20,094 & 2,977 & 436 & 3 & 457,674 & 208,426 &  37 & 714,004 \\
$9 \times 10^5$ & 28,356 & 23,924 & 3,441 & 513 & 9 & 504,779 & 238,632 &  42 & 799,696 \\
\hline
  \end{tabular}
\end{table}

Among all these algorithms, the \textsf{FEMPM} is the most efficient
as well as stable one. The high performance of \textsf{FEMPM} is
mainly due to the \emph{root} node of AMT, which is almost always the
Hash Table as the number of patterns grows. As mentioned before, the
root plays the role of another ``filter'', by which a large number of
positions of the text that can not be filtered by the filter part may
have a chance to be filtered out by the root. Moreover, the increasing
in the number of patterns mainly leads to the growing of the width
rather than the depth of AMT. Therefore, for each position of the
text, the checking time changes slightly. From the experimental
results, we can even give a rough estimate on the matching time based
on the size of pattern set as follows: given the text of 200 MB, an
increase of $10^5$ in the number of patterns will lead to roughly $1$
more second taken in the matching time. The statistics about the node
types of the corresponding AMTs for the small group are shown in Table
\ref{tab:small}. We can see that an increase of $10^5$ in the number
of patterns will yield about $9 \times 10^4$ tree nodes.

\begin{figure}[htbp]
  \centering
  \includegraphics[width=0.6\textwidth]{./eps/large_group}
  \caption{Matching times for the large group.}
  \label{fig:large_group}
\end{figure}

For the large group, the mean matching times of the investigated
algorithms are shown in Figure \ref{fig:large_group}. It can be seen
that, as the number of patterns grows from $10^6$ to $10^7$, the
increases in the matching time of the investigated algorithms are:
$410\%$ (\textsf{WM}), $374\%$ (\textsf{Pre-filter+AC}), $238\%$
(\textsf{MASM}), $112\%$ (\textsf{FEMPM}) respectively. For every
increase of $10^6$ in the number of patterns, the mean increment in
the matching time of the investigated algorithms are: $23.7s$
(\textsf{WM}), $17.0s$ (\textsf{Pre-filter+AC}), $12.3s$
(\textsf{MASM}), $3.5s$ (\textsf{FEMPM}) respectively. This indicates
that, \textsf{Pre-filter+AC} and \textsf{WM} perform not well for
large pattern sets. This is because, as the number of patterns grows,
the chances for every position of the text to successfully match a
pattern increases, which will reduce the effect of the filters of
these algorithms and results in performance degradation. The
\textsf{MASM} algorithm outperforms \textsf{Pre-filter+AC} for pattern
sets whose size are larger than $3 \times 10^6$, which is due to its
pattern set compression strategy and the good scalability of its
pipelinded binary search tree.

\begin{table}[!htp]
  \caption{The statistics about the AMTs for the large group}
  \scriptsize
  \label{tab:large_group}
  \begin{tabular}{rrrrrrrrrr}
 \hline
 Size &
 Single Char &
 Map 4 &
 Map 16 &
 Map 48 &
 Map 256 &
 Single String &
 String Array   &
 Hash Table &
 Total\\
\hline
$1 \times 10^6$ &  38,935 &   27,428  &   6,468 &     851   &    4 &    588,343  &    240,886 &  1,027 &    903,942  \\
$2 \times 10^6$ &  85,900 &   68,973  &  17,084 &   2,141   &   20 &  1,091,527  &    496,001 &  2,015 &  1,763,661  \\
$3 \times 10^6$ & 136,054 &  117,527  &  28,092 &   3,410   &   17 &  1,547,659  &    764,359 &  2,749 &  2,599,867  \\
$4 \times 10^6$ & 189,926 &  171,155  &  40,328 &   4,716   &   23 &  1,966,326  &  1,045,558 &  3,450 &  3,421,482  \\
$5 \times 10^6$ & 247,707 &  230,163  &  83,043 &   5,989   &  38 &  2,354,308   &  1,332,896 &  3,885 &  4,228,029  \\
$6 \times 10^6$ & 304,863 &  293,591  &  65,475 &   7,169   &   49 &  2,706,094  &  1,625,261 &  4,379 &  5,006,881  \\
$7 \times 10^6$ & 366,861 &  361,695  &  77,899 &   8,434   &   53 &  3,030,673  &  1,925,027 &  4,733 &  5,775,380  \\
$8 \times 10^6$ & 429,121 &  433,708  &  90,280 &   9,660   &   62 &  3,336,765  &  2,226,879 &  5,070 &  6,531,545  \\
$9 \times 10^6$ & 494,278 &  509,992  & 102,402 &  10,833   &   72 &  3,623,407  &  2,537,413 &  5,224 &  7,283,621  \\
$1 \times 10^7$ & 558,241 &  591,455  & 115,012 &  11,922   &   79 &  3,986,683  &  2,847,277 &  5,505 &  8,026,174  \\
\hline
\end{tabular}
\end{table}

On the other hand, among all the algorithms, \textsf{FEMPM} is the
most efficient algorithm for large pattern sets. An increase of $10^6$
in the number of patterns, only requires 3 more seconds to match, this
is mainly due to the good scalability of the AMT which enables the
matching time grows very slow as the number of patterns explodes. The
statistical data about the AMTs for the large group are shown in Table
\ref{tab:large_group}. Every $10^6$ increase in the number of patterns
will increase only about $8 \times 10^5$ tree nodes.

\section{Conclusion and Future Works}
\label{sec:conclusion}

In this paper, we propose a Fast Engine for Multi-Pattern Matching
(\textsf{FEMPM}) which includes a filter part and a verification
part. The filter part is used to quickly filter out the positions that
are impossible to match any pattern, while the verification part is
used to verify the potential positions. The verification part used a
data structure called Adaptive Matching Tree (AMT) which can be
constructed to fit the feature of the given pattern set. Owing to the
scalability and compactness of AMT, \textsf{FEMPM} offers a good
support for large-scale pattern sets. Moreover it can improve the
robustness of \textsf{FEMPM} for pattern sets with various $lsp$s. In
the future, we will try to design more efficient inner structures of
the tree nodes.

\section{Acknowledgments}

This work is supported by National Natural Science Foundation of China
(No.61472297) and the Fundamental Research Funds for the Central
Universities (BDZ021430).


\begin{thebibliography}{99}

\bibitem{Smith1981} T.F. Smith and M.S. Waterman, “Identification of
  Common Molecular Subsequences,” J. Molecular Biology, vol. 147,
  pp. 195-197, 1981.

\bibitem{Sankoff1972} D. Sankoff, “Matching Sequences under
Deletion/Insertion Con- straints,” Proc. Nat’l Academy Sciences USA,
vol. 69, pp. 4-6, Jan.  1972.

\bibitem{Hsu1984} W.J. Hsu and M.W. Du, “Computing a Longest Common
  Subsequence for a Set of Strings,” BIT Numerical Math., vol. 24,
  no. 1, pp. 45-59, 1984.

\bibitem{Hirschberg1977} D.S. Hirschberg, “Algorithms for the Longest Common Subse-
quence Problem,” J. ACM, vol. 24, pp. 664-675, 1977

\bibitem{Apostolico1992} A. Apostolico, S. Browne, and C. Guerra,
  “Fast Linear-Space Com- putations of Longest Common Subsequences,”
  Theoretical Com- puter Science, vol. 92, no. 1, pp. 3-17, 1992.

\bibitem{Masek1980} W.J. Masek and M.S. Paterson, “A Faster Algorithm
  Computing String Edit Distances,” J. Computer System Sciences,
  vol. 20, pp. 18- 31, 1980.

\bibitem{Rick1994} C. Rick, “New Algorithms for the Longest Common
  Subsequence Problem,” Technical Report No. 85123-CS, Computer
  Science Dept., Univ. of Bonn, Oct. 1994.
  


\bibitem{pizzachil} {\it http://pizzachil.dcc.uchile.cl/}

\end{thebibliography}  


\vspace*{-0.01in}
%\vspace*{-0.3in}
\noindent
\rule{12.6cm}{.1mm}
pp
\end{document}

%%% Local Variables:
%%% mode: latex
%%% TeX-master: t
%%% End:

%%% Local Variables:
%%% mode: latex
%%% TeX-master: t
%%% End:
